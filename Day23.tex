\documentclass{article}

\usepackage{listings}
\usepackage{graphicx}

\title{Day 23}
\author{Richel Bilderbeek}
\date{\today}

\begin{document}

\maketitle

\begin{abstract}
This article is created from within the CAS program Maxima
and converted to PDF by using FOSS tools only, to assess the viability of this pipeline.As a testcase, the exercise 'Kin selection and the evolution of dispersal' was used,
an obligatory exercise in the Evolutionary Theory Course given in 2014-2015 at the
University of Groningen.
\end{abstract}

\section{Introduction}

\LaTeX~is commonly used for writing publishable scientific articles\cite{gaudeul2006}.
Algebraic manipulations can be done by a CAS, for example Maxima, Maple or Mathematica.
Maxima is the only free and open-source program, and it is the oldest free and open-source computer algebra system, with development started in 1967 (as Macsyma) or 1982 (as MAXIMA).
This article is an example of writing a \LaTeX~ article within Maxima

Writing \LaTeX~ is slower and introduces more errors 
then using Microsoft Word as a text editor \cite{knauff&nejasmic2014}.
In this article, however, the text of a document is generated. \LaTeX~ can
parse that text to create a document, like a Word .docx could do as well,
with less markup (XML) added. Additionally, all tools in the current pipeline
are FOSS, so anyone with an internet connection can access these without
paying licensing costs.

Maxima its capabilities is tested by doing the exercise 'Kin selection and the evolution of dispersal',
an obligatory exercise in the Evolutionary Theory Course given in 2014-2015 at the
University of Groningen.

\section{Exercise}

\begin{table}[here]
  \centering
  \begin{tabular}{ | r | l | }
    \hline
    symbol & description \\
    \hline
    $c$      & cost of dispersal, chance to die when dispersing \\
    $d$      & dispersal rate of mutant \\
    $d\_bar$  & dispersal rate of resident \\
    $d\_star$ & evolutionary stable singularity of dispersal rate \\
    $k$      & probability that an individual present in a patch after dispersal was born there \\
    $n$      & Patch size (=number of females, as haploid) \\
    \hline
  \end{tabular}
  \caption{Definitions}
  \label{table:table_definition}
\end{table}

\subsection{a}

The focal female stays in a patch:

\begin{equation}
{\it N\_stay}\left(d\right)=1-d\label{eq:a_1}
\end{equation}

Her presence will be diluted by unrelated dispersers that enter the patch with rate:

\begin{equation}
{\it N\_in}\left(d\right)=d\left(1-c\right)\label{eq:a_2}
\end{equation}
It might be imagined that a disperser lands in the same patch as it originated in. With an infinite amount of patches, this will never happen.
Therefore, the relatedness k can be concluded to be $k = N\_stay / (N\_stay + N\_in)$, which can be solved for $d$:
\begin{equation}
k\left(d\right)={{1-d}\over{-d+d\left(1-c\right)+1}}\label{eq:a_3}
\end{equation}
\section{Discussion}

Writing \LaTeX~within Maxima can be done, but it is a bit cumbersome:
Maxima does not know \LaTeX~syntax and just creates contextless strings,
which might not be compilable by \LaTeX.
However, because the script does create a .tex file,
this file can be inspected easily with a \LaTeX~tool like texmaker.

\begin{thebibliography}{9}

\bibitem{case2000}
  Case, Ted J.
  2000
  An illustrated guide to theoretical ecology.

\bibitem{gaudeul2006}
  Gaudeul, A.
  2006
  Do Open Source Developers Respond to Competition?: The (La)TeX Case Study.
  Available at SSRN: http://ssrn.com/abstract=908946 or http://dx.doi.org/10.2139/ssrn.908946

\bibitem{knauff&nejasmic2014}
  Knauff, M. and Nejasmic, J.
  December 19, 2014
  An Efficiency Comparison of Document Preparation Systems Used in Academic Research and Development.
  PLoS ONE 9(12): e115069. doi: 10.1371/journal.pone.0115069

\bibitem{otto&day2007}
  Otto, Sarah P. and Day, T.
  2007
  A biologist's guide to mathematical modeling in ecology and evolution.
  ISBN-13: 978-0-691-12344-8

\end{thebibliography}

\appendix

\section{Script file}

\lstinputlisting[language=C++,showstringspaces=false,breaklines=true,frame=single]{Day23.sh}

\section{Maxima file}

\lstinputlisting[language=C++,showstringspaces=false,breaklines=true,frame=single]{Day23.txt}

\section{\LaTeX~file}

\lstinputlisting[language=tex,showstringspaces=false,breaklines=true,frame=single]{Day23.tex}

\end{document}
