\documentclass[11]{article}
\usepackage{graphicx}

\begin{document}








\section{(a) Clutch size}

Clutch size is dependent on multiple variables:

\bigskip

\begin{tabular}{ | r | l | }
  \hline
  Symbol & description \\ 
  \hline
  $a$ & fraction of sons aborted \\
  $C$ & clutch size \\
  $E$ & total energy invested in clutch \\
  $e\_daughter$ & energy investment to produce one healthy daughter \\
  $e\_son$ & energy investment to produce one healthy son \\
  $s$ & primary sex ratio \\
  \hline
\end{tabular}

\bigskip

\input{tmp_C.tex}


\section{(b) Secondary sex ratio}

Secondary sex ratio with no sons aborted:

\input{tmp_S_a_zero.tex}

Secondary sex ratio with all aborted:

\input{tmp_S_a_one.tex}

Secondary sex ratio in general:

\input{tmp_S.tex}

Plotted:

\includegraphics[scale=0.5]{tmp_plot_s}







\section{(c) Shaw-Mohler}

The fitness of a mutant in a resident population is:

\input{tmp_FitnessMutant_1.tex}

To calculate the fitness of a mutant in a resident population, we'll need:

\begin{itemize}
  \item $m(a)$: number of surviving sons
  \item $f(a)$: number of surviving daughters
  \item $v\_male(a\_star)$: male reproductive value in resident population
  \item $v\_female(a\_star)$: female reproductive value in resident population
\end{itemize}

\subsection{(c.1) Surviving sons}

General formula of surviving sons:

\input{tmp_SurvivingSons_1.tex}

Plug in $C(s,a)$:

\input{tmp_SurvivingSons_2.tex}

Set $s=1/2$:

\input{tmp_SurvivingSons_3.tex}

Also plug in $S(a)$:

\input{tmp_SurvivingSons_4.tex}

\subsection{(c.2) Surviving daughters}

General formula of surviving daughters:

\input{tmp_SurvivingDaughters_1.tex}

Plug in $C(s,a)$:

\input{tmp_SurvivingDaughters_2.tex}

Set $s=1/2$:

\input{tmp_SurvivingDaughters_3.tex}

Also plug in $S(a)$:

\input{tmp_SurvivingDaughters_4.tex}

\subsection{(c.3) Male reproductive value}

General formula for male reproductive value:

\input{tmp_ReproValueMale_1.tex}

Filling in $alpha=1/2$:

\input{tmp_ReproValueMale_2.tex}

Filling in $m(a\_star)$:

\input{tmp_ReproValueMale_3.tex}

\subsection{(c.4) Female reproductive value}

General formula for female reproductive value:

\input{tmp_ReproValueFemale_1.tex}

Filling in $alpha=1/2$:

\input{tmp_ReproValueFemale_2.tex}

Filling in $f(a\_star)$:

\input{tmp_ReproValueFemale_3.tex}

\subsection{(c.5) Fitness}

As a reminder, the fitness formula for a mutant in a resident population is:

\input{tmp_FitnessMutant_1.tex}

Plugging in the four functions above:

\input{tmp_FitnessMutant_2.tex}

The fitness of the resident population is: 

\input{tmp_FitnessResident_1.tex}

Plugging in the four functions (with $a=a\_star$) above, this simplifies to:

\input{tmp_FitnessResident_2.tex}

The relative fitness of the mutant on the resident population:

\input{tmp_RelativeFitness_1.tex}

Filling in both fitness functions:

\input{tmp_RelativeFitness_2.tex}

For an ESS, the fitness of a mutant must always be lower in a resident population. To calculate this, we need the partial derivative of the above fitness function to $a$.

\input{tmp_DeltaFitness_1.tex}

Which is equal to:

\input{tmp_DeltaFitness_2.tex}

Filling in $e_son=10$ and $e_daughter=10$:

\input{tmp_Ess_1.tex}

Which is equal to:

\input{tmp_Ess_2.tex}






\end{document}


