\documentclass{article}

\usepackage{listings}
\usepackage{graphicx}

\title{Day 1, exercise 4: Vigilance}
\author{Richel Bilderbeek}
\date{\today}

\begin{document}

\maketitle

$v$: fraction of foraging time invested in being watchful.\\
$S$: survival probability.
$$S\left(v\right)=v$$
$F$: foraging efficiency.
$$F\left(v\right)=1.0-v^2$$
$W(v)$: fitness.
$W(v) = S(v) + F(v)$
$$W\left(v\right)=-v^2+v+1.0$$
Plotted this looks like:\\\\
\fbox{\includegraphics[scale=0.5]{/home/richel/GitHubs/Maxima/Day1_4_vigilance_output.pdf}}\\\\
To calculate the maximum or minimum, set the derivate to zero and solve it:
$${{d}\over{d\,v}}\,W\left(v\right)=1-2\,v=0$$
$$v={{1}\over{2}}$$
Thus, the optimal vigilance level $v$ equals:$${{1}\over{2}}$$

This optimal vigilance level results in a fitness of:$$W\left({{1}\over{2}}\right)=1.25$$

To find out if it is a fitness minimum or maximum,
calculate the second derivative
and find out its value at the minimum or maximum:
$${{d^2}\over{d\,v^2}}\,W\left(v\right)=-2$$
Thus, it is a maximum.

\appendix

\section{Script file}

\lstinputlisting[language=C++,showstringspaces=false,breaklines=true,frame=single]{Day1_4_vigilance.sh}

\section{Maxima file}

\lstinputlisting[language=C++,showstringspaces=false,breaklines=true,frame=single]{Day1_4_vigilance.txt}

\section{\LaTeX~file}

\lstinputlisting[language=tex,showstringspaces=false,breaklines=true,frame=single]{Day1_4_vigilance_output.tex}

\end{document}
